\documentclass[../main.tex]{subfiles}

\begin{document}

\subsection{강의 계획}
본 교재에는 Swift 5의 기본적인 문법을 빠르게 익힌 후, 객체 지향 프로그래밍의
개념에 익숙해지도록 도와줍니다.
나아가 본 수업을 통해 Swift와 Cocoa Touch 프레임워크를 사용해 iOS 모바일
어플리케이션을 개발하는 방법에 대해 배웁니다.
기존에 Python이나 C 언어 등 다른 언어로 어느 정도는 프로그래밍을 해보았다고
가정합니다.
다음은 Swift로 \texttt{Hello, World!}를 출력하는 코드입니다:
\begin{minted}[mathescape,
               linenos,
               breaklines,
               numbersep=5pt,
               frame=lines,
               framesep=2mm]{swift}
print("Hello, World!")
\end{minted}
이는 Python 3와 작성한 Hello World 프로그램과 정확히 일치합니다.
이처럼 Swift는 모던 언어이기 때문에 문법적으로 간결하고, 이미 프로그래밍을
해보았다면 하루 정도면 어느 정도 원하는 작업을 수행하는 코드를 작성할 수 있을
것입니다.
그렇지만 본인이 원래 알던 언어의 억양을 강하게 가진 상태로 Swift 코드를 짜는
것은 그리 바람직하지 못한 일이므로, Swift의 기본적인 문법을 알아봅시다.

\subsection{Swift의 기본 요소}
\subsubsection{상수Constant와 변수Variable}
Swift에서는 상수와 변수의 구분을 중요하게 여깁니다.
\begin{minted}[mathescape,
               linenos,
               breaklines,
               numbersep=5pt,
               frame=lines,
               framesep=2mm]{swift}
var myVariable = 10
myVariable = 20
let myConstant = 7
myConstant = 8  // Cannot assign to value: 'myConstant' is a 'let' constant
\end{minted}

Swift에서도 Python과 마찬가지로 손쉽게 변수 swapping을 할 수 있습니다.
이 역시 multiple assignment를 사용한 것으로, 아래와 같이 사용할 수 있습니다.
\begin{minted}[mathescape,
               linenos,
               breaklines,
               numbersep=5pt,
               frame=lines,
               framesep=2mm]{swift}
var (a, b, c) = (2, 7, 12)
(a, b, c) = (b, c, a)
print(a, b, c)  // 7 12 2
(a, b) = (b % a, a)
print(a, b)  // 5 7
\end{minted}

Swift에서도 Python과 같이 모든 사칙 연산이 가능하지만, 지수 연산자
\texttt{**}에 해당하는 연산자는 존재하지 않습니다.
대신 \texttt{Foundation} 프레임워크에 내장되어 있는 \texttt{pow(\_:\_:)}를
사용할 수 있습니다.

\texttt{Foundation} 프레임워크는 Swift의 표준 라이브러리에 추가적으로 사용할 수
있는 프레임워크입니다.
앞으로 iOS 앱 개발에서 GUI를 다룰 때 항상 코드 맨 앞에 \texttt{UIKit}을 불러올 
것인데, 이는 \texttt{Foundation}을 이미 포함하는 것입니다.
따라서 UI 와 무관한 기본적인 기능을 구현할 때에는 보통 \texttt{Foundation}을
불러와 구현합니다.
예를 들어, Swift는 언어 자체에는 \texttt{URL}이라는 자료형이 없지만,
\texttt{Foundation}을 사용하면 이를 쉽게 다룰 수 있습니다.

Swift, C, Java 등은 정적 타입 언어이고, Python, JavaScript, Ruby 등은 동적 타입
언어입니다.
무슨 말이냐 하면, Swift와 같은 정적 타입 언어는 컴파일 시 어떤 값의 자료형이
정해지고, Python과 같은 동적 타입 언어는 런타임 시 값의 자료형이 바뀔 수
있습니다.
즉, 아래와 같은 코드는 Swift에서 허용되지 않습니다.
\begin{minted}[mathescape,
               linenos,
               breaklines,
               numbersep=5pt,
               frame=lines,
               framesep=2mm]{swift}
var myInt = 4
myInt = 1.2  // Cannot assign value of type 'Double' to type 'Int'
\end{minted}
실제로 위에서 첫번째 줄은 \texttt{var myInt: Int = 4}를 줄여 쓴 것입니다.
C에서처럼 \texttt{int myInt = 4}으로 자료형을 항상 명시할 필요는 없지만,
강조하거나 컴파일러가 자료형을 추측할 수 없을 경우 표시해주어야 합니다.
또한 특정적인 점은 자료형을 변수 뒤에 써준 다는 것입니다.
이는 새로운 것은 아니며 Pascal의 방식을 따른 것입니다.

Swift에서는 문자를 다루는 것이 Python만큼 간결하지는 않지만, 간단한 부분에
있어서는 간편하게 쓸 수 있습니다.
Swift에서는 문자열을 큰따옴포(\texttt{"})로만 감쌀 수 있습니다.
작은 따옴표는 허용되지 않습니다.
또한 문자열에서 다른 변수의 값을 넣는 것이 매우 간편합니다:
\begin{minted}[mathescape,
               linenos,
               breaklines,
               numbersep=5pt,
               frame=lines,
               framesep=2mm]{swift}
let value = 4
print("value is \(value).")
\end{minted}
와 같이 \texttt{\textbackslash($\cdot$)}안에 변수를 감싸면 됩니다.

배열을 만드는 것도 Python만큼이나 Swift에서도 매우 간단합니다.
똑같이 \texttt{[$\cdot$]}으로 감싸고, 인덱스로 접근하는 방식도 같습니다:
\begin{minted}[mathescape,
               linenos,
               breaklines,
               numbersep=5pt,
               frame=lines,
               framesep=2mm]{swift}
var myList = ["Swift", "Python", "C", "Objective-C", "Scala", "Haskell"]
print(myList[0])
\end{minted}
단, Swift가 정적 타입 언어라는 것을 유념해야 합니다.
Python처럼 아무 값이나 전부 배열로 만들 수 없으며, 반드시 한 종류의 자료형만을
담고 있어야 합니다.
즉, 위에서 \texttt{myList}는 \texttt{[String]}의 자료형을 가지고 있습니다.

이는 딕셔너리도 마찬가지입니다:
\begin{minted}[mathescape,
               linenos,
               breaklines,
               numbersep=5pt,
               frame=lines,
               framesep=2mm]{swift}
var myDictionary = ["Swift": 2014,
                    "Python": 1991,
                    "C": 1972,
                    "Objective-C": 1984,
                    "Scala": 2004,
                    "Haskell": 1990]
myDictionary[1996] = "OCaml"
// Cannot subscript a value of type '[String : Int]' with an index of type 'Int'
myDictionary["Kotlin"] = "2011"
// Cannot assign value of type 'String' to type 'Int?'
\end{minted}

빈 배열이나 딕셔너리의 경우, Swift가 자료형을 추측할 수 있도록 다음과 같이
정의할 수 있습니다:
\begin{minted}[mathescape,
               linenos,
               breaklines,
               numbersep=5pt,
               frame=lines,
               framesep=2mm]{swift}
var myList1: [Int] = []
var myList2 = [Float]()
var myList3 = []  // Empty collection literal requires an explicit type

var myDictionary1: [String: Bool] = [:]
var myDictionary2 = [Double: UInt]()
var myDictionary3 = [:]  // Empty collection literal requires an explicit type
\end{minted}

조건문 \texttt{if}입니다:
\begin{minted}[mathescape,
               linenos,
               breaklines,
               numbersep=5pt,
               frame=lines,
               framesep=2mm]{swift}
let grades = ["A+", "A0", "B+", "A+", "A-", "F"]
if "A+" in grades {
    print("A+ is in grades!")
} else {
    print("A+ is not in grades!")
}
\end{minted}
Python과 다른 점은 콜론(\texttt{:}) 대신 중괄호(\texttt{\{$\cdot$\}})를
사용하여 구문을 구분한다는 점입니다.

반복문 \texttt{for}입니다:
\begin{minted}[mathescape,
               linenos,
               breaklines,
               numbersep=5pt,
               frame=lines,
               framesep=2mm]{swift}
let grades = ["A+", "A0", "B+", "A+", "A-", "F"]
for grade in grades {
    print(grade)
}
\end{minted}
혹은, 다음과 같이 클로져와 함께 \texttt{forEach(\_:)} 메소드를 사용할 수
있습니다:
\begin{minted}[mathescape,
               linenos,
               breaklines,
               numbersep=5pt,
               frame=lines,
               framesep=2mm]{swift}
let grades = ["A+", "A0", "B+", "A+", "A-", "F"]
grades.forEach { grade in
    print(grade)
}
\end{minted}
혹은 더 짧게:
\begin{minted}[mathescape,
               linenos,
               breaklines,
               numbersep=5pt,
               frame=lines,
               framesep=2mm]{swift}
let grades = ["A+", "A0", "B+", "A+", "A-", "F"]
grades.forEach { print($0) }
\end{minted}
클로져에 대해서는 뒤에 함수 부분에서 다시 소개하도록 하겠습니다.
클로져는 \textit{매우} 중요한 내용입니다.

Python에 없었던 내용 중에는 \texttt{switch} 문이 있습니다.
앱에서 다양한 경우를 고려해야 할 경우, \texttt{switch}를 사용하면 편리합니다.
\begin{minted}[mathescape,
               linenos,
               breaklines,
               numbersep=5pt,
               frame=lines,
               framesep=2mm]{swift}
let test = "sample"
switch test {
case "wow":
    print("wow case")
case "foo", "bar":
    print("foo-bar case")
case let x where x.hasPrefix("sample"):
    print("sample")
default:
    print("default")
}
// sample
\end{minted}
C와의 비교했을 때 눈에 띄는 차이점은 \texttt{break}가 없다는 것으로, Swift의
\texttt{switch} 구문은 \texttt{case}에서 다음으로 fall through하지 않습니다.
만약 C처럼 fall through가 필요하다면 \texttt{fallthrough} 키워드를 사용하면
됩니다.
또한 \texttt{let}을 통한 조건 확인을 통해서 매우 강력한 도구가 될 수 있습니다.
나아가 enum의 associated value를 활용한다면 더욱 빛을 발합니다.

\texttt{while}과 \texttt{repeat}-\texttt{while} 구문 설명하기

함수와 클로져 설명하기 (\texttt{map}, Python 람다...)

\end{document}
